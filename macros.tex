\newcommand{\minus}{{\scalebox{0.9}{-}}}
\newcommand{\plus}{{\scalebox{0.6}{\!+}}}
\newcommand{\squarediamond}{\mathbin{\text{\tikz [x=1ex,y=1ex,scale=1.5,line width=.1ex,line join=round] \draw (0,0) rectangle (1,1) (.5\pgflinewidth,.5) -- (.5,1ex-.5\pgflinewidth) -- (1ex-.5\pgflinewidth,.5) -- (.5,.5\pgflinewidth) -- (.5\pgflinewidth,.5) -- cycle;}}}
\renewcommand{\square}{\mathbin{\text{\tikz [x=1ex,y=1ex,scale=1.5,line width=.1ex,line join=round] \draw (0,0) rectangle (1,1);}}}
\renewcommand{\diamond}{\mathbin{\text{\tikz [x=1ex,y=1ex,scale=1.5,line width=.1ex,line join=round] \draw (.5\pgflinewidth,.5) -- (.5,1ex-.5\pgflinewidth) -- (1ex-.5\pgflinewidth,.5) -- (.5,.5\pgflinewidth) -- (.5\pgflinewidth,.5) -- cycle;}}}
\newcommand{\state}[1]{\ensuremath \langle #1 \rangle}
\newcommand{\transition}[2]{\ensuremath \state{#1} & \longrightarrow \state{#2}}
\newcommand{\termrule}[3]{\ensuremath #1 \xrightarrow{#2} #3}
\newcommand{\fail}{\ensuremath \text{$\uparrow$}}
\newcommand{\success}{\ensuremath \text{$\downarrow$}}
\newcommand{\cont}[2]{\ensuremath \mathbf{#1}(#2)}
\newcommand{\return}{\ensuremath \hookleftarrow}
\newcommand{\seq}[2]{\ensuremath \operatorname{seq}(#1,#2)}
\newcommand{\choice}[2]{\ensuremath \operatorname{choice}(#1,#2)}
\newcommand{\leftchoice}[2]{\ensuremath \operatorname{lchoice}(#1,#2)}
\newcommand{\fix}[2]{\ensuremath \operatorname{rec}(#1,#2)}
\newcommand{\path}[2]{\ensuremath \operatorname{path}(#1,#2)}
\newcommand{\one}[1]{\ensuremath \operatorname{one}(#1)}
\newcommand{\all}[1]{\ensuremath \operatorname{all}(#1)}
\newcommand{\some}[1]{\ensuremath \operatorname{some}(#1)}
\newcommand{\congr}[1]{\ensuremath \operatorname{cong}(#1)}
\newcommand{\match}[1]{\ensuremath \operatorname{match}(#1)}
\newcommand{\build}[1]{\ensuremath \operatorname{build}(#1)}
\newcommand{\where}[1]{\ensuremath \operatorname{where}(#1)}
\newcommand{\test}[1]{\ensuremath \operatorname{test}(#1)}
\newcommand{\negate}[1]{\ensuremath \operatorname{neg}(#1)}
\newcommand{\scope}[2]{\ensuremath \operatorname{scope}(#1,#2)}
\newcommand{\vars}[1]{\ensuremath \text{vars}(#1)}
\newcommand{\alt}{\ensuremath \; | \;}
\newcommand{\subst}[3]{\ensuremath #1[#2 / #3]}
\newcommand{\transform}[5]{#1, #2 \xrightarrow{#3} #4, #5}
\newcommand{\transformx}[4]{#1, #2 \xrightarrow{#3} #4}
\newcommand{\transformxy}[3]{#1 \xrightarrow{#2} #3}
\newcommand{\transformfail}[3]{#1, #2 \xrightarrow{#3} \fail}
\newcommand{\dom}{\ensuremath \operatorname{dom}}
\newcommand{\cod}{\ensuremath \operatorname{cod}}
\newcommand{\cjm}{\ensuremath \xmapsto{\text{\tiny cjm}}}
\newcommand{\mon}{\ensuremath \xmapsto{\text{\tiny mon}}}
\newcommand{\Strat}{\ensuremath \Cat{Strat}}
\newcommand{\Term}{\ensuremath \mathbf{Term}}
\newcommand{\Fail}{\ensuremath \text{$\Uparrow$}}
\newcommand{\Success}{\ensuremath \text{$\Downarrow$}}
\newcommand{\Var}{\ensuremath \mathbf{Var}}
\newcommand{\Constructor}{\ensuremath \Cat{Con}}
\newcommand{\Env}{\ensuremath \Var \mapsto \Term}
\newcommand{\Pow}{\ensuremath \mathscr{P}}
\newcommand{\sem}[1]{\ensuremath \lBrack #1 \rBrack}
\newcommand{\semn}[1]{\ensuremath \lBrack #1 \rBrack_{\mathcal{N}}}
\newcommand{\semsalg}[1]{\ensuremath \lBrack #1 \rBrack_{\SAlg}}
\newcommand{\alphan}{\ensuremath \alpha_\Naive}
\newcommand{\Disc}[1]{\ensuremath \left\vert #1 \right\vert}
\newcommand{\Cat}[1]{\ensuremath \mathbf{#1}}
\newcommand{\Nat}{\ensuremath \mathbb{N}}
\newcommand{\Hom}[1]{\ensuremath #1}
\newcommand{\twoHom}[1]{\ensuremath #1}
\newcommand{\State}{\ensuremath \mathbf{\Sigma} }
\newcommand{\Statea}{\ensuremath \widehat{\State}}
\newcommand{\setbuild}[2]{\ensuremath \{\, #1 \mid #2 \,\}}
\newcommand{\setbuildc}[1]{\ensuremath \{\, #1 \,\}}
\newcommand{\id}{\ensuremath \operatorname{id}}
\newcommand{\Id}{\ensuremath \operatorname{Id}}
\newcommand{\Rel}{\ensuremath \Cat{Rel}}
\newcommand{\Set}{\ensuremath \Cat{Set}}
\newcommand{\Naive}{\ensuremath \Cat{Naive}}
\newcommand{\adjoint}{\ensuremath \dashv}
\newcommand{\lub}{\ensuremath \bigsqcup}
\newcommand{\glb}{\ensuremath \bigsqcap}
\newcommand{\SAlgebra}{\ensuremath \mathcal{S}\text{-algebra}}
\newcommand{\SAlgebras}{\ensuremath \mathcal{S}\text{-algebras}}
\newcommand{\SAlg}{\ensuremath \Cat{SAlg}}
\newcommand{\domain}{\ensuremath \operatorname{domain}}
\newcommand{\lfail}{\ensuremath \operatorname{fail}}
\newcommand{\lsucc}{\ensuremath \operatorname{succ}}
\newcommand{\get}{\ensuremath \operatorname{get}}
\newcommand{\map}{\ensuremath \operatorname{map}}
\newcommand{\arity}{\ensuremath \operatorname{arity}}
\newcommand{\arityeq}{\ensuremath \operatorname{arityeq}}
\newcommand{\aritylt}{\ensuremath \operatorname{aritylt}}
\newcommand{\arityneq}{\ensuremath \operatorname{arityneq}}
\newcommand{\coneq}{\ensuremath \operatorname{coneq}}
\newcommand{\conneq}{\ensuremath \operatorname{conneq}}
\newcommand{\ttop}{\ensuremath \top_{\Term}}
\newcommand{\tbot}{\ensuremath \bot_{\Term}}
\newcommand{\comp}{\ensuremath \mathrel{\circ}}
\newcommand{\twocomp}{\ensuremath \mathrel{\circledwhitebullet}}
\newcommand{\Comp}{\ensuremath \mathop{\bigcirc}}
\newcommand{\witharity}{\ensuremath \Join}
\newcommand{\iso}{\ensuremath \cong}
\newcommand{\lfix}{\ensuremath \operatorname{rec}}
\newcommand{\FAlg}{\ensuremath F\text{-Alg}}
\newcommand{\TwoAdj}{\ensuremath 2\text{-Adj}}
\newcommand{\sendto}[2]{ \{ #1 \} \rightarrow \{ #2 \} }
\newcommand{\asendto}[2]{ #1 \rightarrow #2 }
\newcommand{\ontopof}[2]{ \begin{array}{c} #1 \\ #2 \end{array} }
\newcommand{\eq}[1]{\ensuremath \operatorname{eq}_{#1}}
\newcommand{\injl}{\ensuremath \operatorname{inj}_1}

%%% Local Variables:
%%% mode: latex
%%% TeX-master: "system-s"
%%% End:
